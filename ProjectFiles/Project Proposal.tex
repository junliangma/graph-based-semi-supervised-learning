\documentclass[]{article}
\usepackage{graphicx}
\usepackage{textcomp}
\usepackage{listings}
\usepackage{tabulary}
\usepackage{titling}
\usepackage{natbib}
\usepackage{lipsum}



%opening
\title{Empirical evaluation of Graph Based Semi Supervised Learning}
\date{}
\let\OLDthebibliography\thebibliography
\renewcommand\thebibliography[1]{
  \OLDthebibliography{#1}
  \setlength{\parskip}{0pt}
  \setlength{\itemsep}{0pt plus 0.3ex}
}
\begin{document}
\nocite{*}
\author{Deepak Rishi: ID 20605072}
\maketitle
\vspace{-60.6pt}

\section*{$\bullet$ Project Description}
The purpose of this project is to explore the area of Semi Supervised Learning and build Machine Learning Models that make use of unlabelled Data.
Semi-supervised learning falls between unsupervised learning (without any labeled training data) and supervised learning (with completely labeled training data). Unlabeled data, when used in conjunction with a small amount of labeled data, can produce considerable improvement in learning accuracy.

In this project I will be focussing on Graph Based Semi Supervised Learning.
Graph-based methods for semi-supervised learning use a graph representation of the data, with a node for each labeled and unlabeled example. The graph may be constructed using domain knowledge or similarity of examples. I will be comparing different methods to construct the graph and try differnt oprimization functions such as harmonic functions and manifold regularization. Then I will try and compare my results with other semi supervised techniques such as EM and  Transductive SVM.


\bibliographystyle{plain}
\bibliography{bibtext.bib}


\end{document}
